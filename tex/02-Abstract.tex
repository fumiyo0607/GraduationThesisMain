\chapter*{内容梗概}
\pagenumbering{roman}
%%%%%%%%%%%%%%%%%%%%%%%%%
近年ではEC市場規模の拡大に伴い,
各ユーザの購買行動を分析に基づく効果的なマーケティング施策の重要性が高まってきている.
そこで,機械学習の分野においては,購買行動から各ユーザの嗜好を学習し,
ユーザの嗜好に合致したアイテムを推薦するモデルの研究が多く行われている.
ユーザの嗜好を捉える際,購買したアイテムのみでなく,アイテムに関する補助情報を考慮することが必要であると考えられ,
補助情報を取り入れた推薦システムが多く用いられている.
しかし,従来の多くの手法ではその補助情報間の既知の関係性を考慮してモデルが構築されていないため,
アイテムと補助情報間の高次の関係性を捉えることが困難である.
このような中,Deep Learning 手法の1つとしてユーザ,アイテムおよびその補助情報の高次の関係性をグラフ状のデータ構造で表す手法
Knowledge Graph Attention Network(以下,KGAT)が存在する.
このモデルではグラフを用いて,個別のユーザに対するアイテムの推薦理由について解釈可能であり,
従来のモデルの問題点であったデータ間の高次の関係性の表現に加えて,Deep Learing を用いる問題点であった
解釈性を解決している.

しかし,マーケティングの分野では,
ユーザの嗜好は同一の属性をもつユーザである程度類似しているという前提に基づき,
ユーザを属性情報を用いてセグメンテーションし,セグメントごとに施策を実施する(以下,マスマーケティング)ことが多い.
そのため,ユーザ属性を用いたセグメントに対する解釈が求められており,この問題を解決する手法の提案が望まれる.

そこで本研究では,ユーザの嗜好が,「ユーザの属性により説明される嗜好」と「ユーザ固有の嗜好」から構成されると仮定し,
各ユーザの嗜好における属性情報の影響度を定量化する手法について研究を行い,KGATを用いてユーザの嗜好をユーザの属性情報と購買アイテムからモデル化する手法を提案した.
この手法を用いることにより,属性情報の影響度が高いユーザに対してはマスマーケティングを実施し,
影響度が低いユーザに対しては,ユーザにパーソナライズした施策を実施するなど,
より効果的なマーケティング戦略を取ることが可能になると考えられる.
最後に,提案手法の有用性について検討するために,実際のEC市場の評価値データを分析に適用し,得られる結果について考察を与えた.

\tableofcontents

