\chapter{結論と今後の課題}

%%%%%%%%%%%%%%%%%%%%%%%%%%%%%%%%%%%%%%%%%
\section{結論}
%%%%%%%%%%%%%%%%%%%%%%%%%%%%%%%%%%%%%%%%%
\label{sec:結論}
本研究では,SemiBoostの偏りのあるデータに強い性質を利用し,多値分類問題に拡張することでSemiBoostの多値分類への拡張モデルを提案した.その際,直接ECOC法にSemiBoostを適用した場合に良い識別境界を得ることができない可能性がある.その問題点を解決するため,SemiBoostによってラベルなしデータに与えた仮ラベルの信頼度の評価を行い,その信頼度の高いもののみにラベルを付与した.このラベルを付与したデータとラベルありデータを教師あり学習に用いることで,識別境界の悪化を避けつつ,SemiBoostによってラベルを付与したデータによりラベルありデータのみの場合と比較した精度の改善を図った.\par
また,UCIデータセットを用いることで偏りのあるデータに対して提案モデルを適用する有効性を示した.また,ラベルの付与しやすいデータと付与しづらいデータから分類の難しさも分かった.偏りのないランダムサンプルを行ったデータで実験を行った場合,ラベルありデータと同程度の精度を示すことから,データの偏りに関わらず提案手法を利用できることを示すことができた.本研究の成果により,多値分類問題に対しても,精度の良い半教師ありブースティング学習が可能となり,より良い対象問題への応用が期待できる.\par

%%%%%%%%%%%%%%%%%%%%%%%%%%%%%%%%%%%%%%%%%
\section{今後の課題}
%%%%%%%%%%%%%%%%%%%%%%%%%%%%%%%%%%%%%%%%%
\label{sec:今後の課題}
今後の課題として,以下の2点があげられる.\par
全てのデータにラベルがある場合の精度へ近づけることが考えられる.これはラベルなしデータにラベルを付与する精度を上げることとラベルを付与するデータを増やすことが考えられる.\par
カテゴリ数が未知の場合にも対応した半教師あり多値分類手法への拡張などが挙げられる.今回の提案手法ではECOC法を用いたがカテゴリ数が未知の場合は符号表を用いることができなくなるため,全く異なる提案が必要になる.分類方法としては,クラスタリングを利用しつつ分類を行うことで,あるカテゴリに所属するか否かを判断するモデルの作成が考えられる.\par
