\chapter{序論}
\pagenumbering{arabic}

\section{研究背景}
近年EC市場規模の拡大に伴い,ユーザの購買行動に関する膨大なデータが蓄積されている.
この蓄積されたデータからユーザの購買行動を分析し,効果的なマーケティング施策を実施する重要性が高まっている.
そこで,機械学習の分野においては,購買行動から各ユーザの嗜好を学習し,
ユーザの嗜好に合致したアイテムを推薦するモデルの研究が多く行われている.
このように,ユーザ個人に対して異なる施策を実施する方法をマーティングの分野においては,One to One マーケティングといい,情報通信技術の発展に伴い注目を集めている.
最適なOne to One マーケティングを実施するためには,ユーザ個人の嗜好を正確に捉える必要がある.
ユーザの嗜好を正確に捉えるためには,購買したアイテムのみでなく,アイテムに関する補助情報を考慮することが必要であると考えられ,現在では補助情報を取り入れた推薦システムが多く用いられている\cite{rendle2010factorization}.
しかし,従来の多くの手法ではその補助情報間の既知の関係性を考慮してモデルが構築されていないため,
アイテムと補助情報間の高次の関係性を捉えることが困難であり,複雑な関係性をデータからうまく学習することができない.
さらに最近では,Deep Learning を用いた推薦システムがその精度とモデルの柔軟性から評価され,多くの研究が行われている\cite{2019DeepRecSurvey}.しかし,既存のDeep Learning 手法の多くは,モデルに対する解釈を与えることが困難であり,モデルの学習によって得られた結果をアイテムの推薦以外に活かすことができないという問題点があった.
このような中,Deep Learning 手法の1つとしてユーザ,アイテムおよびその補助情報の高次の関係性をグラフ状のデータ構造で表す手法
Knowledge Graph Attention Network(以下,KGAT)\cite{KGAT19}が存在する.
このモデルでは,補助情報を含んだデータからなるグラフをモデルの入力とし,
既知の関係性を捉えることで,最新のDeep Learning を用いた手法からその精度を向上させている.
さらにグラフを用いて,個別のユーザに対するアイテムの推薦理由について解釈可能であり,
従来のモデルの問題点であったデータ間の高次の関係性の表現に加えて,Deep Learning を用いる問題点であった解釈性を与えることに成功している.
このように,より良いOne to One マーケティングを実現するための推薦システムに関する研究が盛んに行われており,様々なモデルが提案されている.

しかし,実際の企業では依然として,ユーザを属性情報を用いてセグメンテーションし,セグメントごとに施策を実施する(以下,セグメントマーケティング)ことが多い.セグメントマーケティング では,One to One マーケティング と異なり多くのユーザを対象とするため,企業のブランディングに繋がり,現在でも重要視されている.
しかし,セグメントマーケティングでは,ユーザの嗜好は同一の属性をもつユーザである程度類似しているという前提に基づいている.
ユーザの嗜好が多様化している昨今では,同一セグメント内でもその嗜好にはバラつきが生じており,セグメント毎に画一的な施策を実施することではニーズに応えることができないユーザが存在することが考えられる.
さらに,近年ではユーザが興味がないと感じたネット上の広告を表示しないようにすることが可能で,画一的な施策を行う際の懸念事項となっている.
そのため,ユーザの嗜好が多様化している現在ではOne to One マーケティングとセグメントマーケティングを組み合わせることにより,効果的なマーケティングを行うことが必要であること考えられ,これを実現するための手法の提案が望まれる.


\section{研究目的}
本研究では,より効果的なマーケティング施策の立案を可能にするために,ユーザの嗜好が,「ユーザの属性により説明される嗜好」と「ユーザ固有の嗜好」から構成されると仮定し,
各ユーザの嗜好における属性情報の影響度を定量化することを考える.
これにより,属性情報の影響度が高いユーザに対してはセグメントマーケティングを実施し,
影響度が低いユーザに対しては,One to One マーケティングを実施するといった戦略を取ることが可能になると考えられる.
% TODO:なぜこの戦略が効果的であるか述べる
% 適切な人にブランディングを実行し,そうでない人には不要な広告を表示させなくていい
具体的には,ユーザとその属性情報,アイテムとその補助情報からなるグラフに対してKGATを適用することで,ユーザの嗜好をユーザの属性情報と購買アイテムからモデル化する手法を提案する.これにより,ユーザの嗜好を属性と購買アイテムの双方の観点からの分析が可能となる.
最後に,提案手法を用いて,実際のECサイトの購買履歴データを分析し,
提案モデルの有用性を示す.


\section{論文の構成}
本論文は,7章から構成されており,以下にその概要を示す.\par
第2章では,本研究で研究対象とするECサイトの概要や特徴,
対象とする問題について述べる.
\par
第3章では,近年の推薦システムの研究動向について述べる.
\par
第4章では,従来手法の研究背景やその概要について述べる.
\par
第5章では,提案モデルの概要,および構築手順について述べる.
\par
第6章では,提案モデルを用いたECサイトの評価値データを用いた分析の概要と結果,および考察を述べる.
\par
第7章では,本研究全体についての考察を述べる.
\par
第8章では,本研究のまとめと今後の課題を述べる.


