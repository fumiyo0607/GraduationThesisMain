\chapter{考察}
本章では,本研究全体を通した考察を述べる.
\section{分類精度}
%\begin{flushleft}
%  \large{\bfseries {}}
%\end{flushleft}\par
第5章の結果より,偏りのあるデータに対して提案手法の有効性が示された.直接多値分類を行うことによる分類器の精度の悪化を引き起こすSemiBoostの欠点を打ち消しつつ,SemiBoostの偏りのあるデータに対して強い性質を活用することができた.その結果,半教師あり学習の問題点であるラベルありデータの偏りに影響をされない分類を行う提案の有効性を示すことができた.\par
また,人工データの実験結果から偏りのあるデータの位置によって分類の難しさが変化することが分かった.提案手法において,比較的分類しやすいカテゴリの分類がされやすくなっている.これはラベル付けされたデータが多いカテゴリに見られる傾向であり,その結果が分類精度に影響したと考えられる.\par

実験全体を通して偏りのあるデータを扱う際,SVMが分類精度に大きく影響しているような傾向が見られた.これは実験を行う際のデータの適正が考えられるため,SVMによる分類と決定木(本研究では,RandomForest)による分類のどちらが良いかは分からないのでデータに合わせて分類器を決める必要があると考えられる.特徴量に依存しそうである場合には,決定木の方が精度が良いと考えられる.SVMは,ラベルを付与したものに誤分類したものがあると精度が悪化する可能性がある.\par


\section{ラベルの付与について}
ラベルなしデータに対して付与するラベルの正確さは分類の難しさによって変化するため,ハミング距離が0となるデータのみを抽出する復号方法を用いても正確なラベルの付与が行えなかった.また,この手法で復号を行うとカテゴリ数が増えるほどハミング距離が0となるラベルを付与できるデータが少なくなるため,カテゴリ数が多いときにはこの手法の効果が無くなることがある.\par
ラベルを付与しやすいカテゴリとそうでないカテゴリが存在し,全データのうち他のカテゴリと距離が遠いカテゴリほどラベルを付与しやすいことが分かった.一方で,他のカテゴリと距離が近いデータにはラベルを付与しにくい傾向があり,複数のカテゴリをまとめて分類を行う提案手法では改善が難しいと考えられる.\par
SemiBoostにおいて,仮ラベルから学習データに用いるデータを経験的に$10\%$としていたが,分類器の作成において学習に用いるデータ数を検討するべきである.これは弱学習器の学習が複雑になるため,少しでも正しい分類を行えるようにしたい意図がある.\par

\section{計算コスト}
提案手法ではSemiBoostを用いたラベルの付与を行う際,カテゴリ数と同数の1-versus-the rest法の判別器で学習を行う必要がある.このときの学習は分散処理を行うことができるため,提案手法での学習時間はSemiBoostによる二つにまとめたカテゴリの分類時間と,その後の教師あり分類器の学習時間となる.したがって,カテゴリ数が増加したとしても学習時間に影響は少なく,全データ数の学習により計算時間が決まると考えられる.この手法において計算量が多い点はSemiBoostのデータ間類似度を測定する部分が主な点である.これはグラフベースな手法の問題点でもあり,計算量の削減も重要な課題である.

