\documentclass[a4j,12pt,openany]{jreport}

\usepackage{amsmath,amssymb,epsbox}
\usepackage{hhline}
\usepackage[dvipdfmx]{graphicx}
\usepackage{ascmac}
\usepackage{listings}
\usepackage{bm}
\usepackage{amsmath}
\usepackage{here}
\usepackage{multicol}
\usepackage{lscape}
\usepackage{remreset}
\usepackage{setspace}
\usepackage{comment}
\usepackage{plext}
\usepackage{booktabs}
\usepackage{multirow}
% \usepackage{listings, jlisting}
\usepackage{color}
\usepackage[dvipdfmx]{graphicx}
\usepackage{slashbox}
%\usepackage {diagbox}


\makeatletter
\def\Hline{%
\noalign{\ifnum0=`}\fi\hrule \@height 2pt \futurelet
\reserved@a\@xhline}
\makeatother




\lstset{
  breaklines = true,
  language=Python,
  basicstyle=\ttfamily\scriptsize,
  commentstyle={\itshape \color[cmyk]{1,0.4,1,0}},
  classoffset=1,
  keywordstyle={\bfseries \color[cmyk]{0,1,0,0}},
  stringstyle={\ttfamily \color[rgb]{0,0,1}},
  frame=tRBl,
  framesep=5pt,
  showstringspaces=false,
  numbers=left,
  stepnumber=1,
  numberstyle=\tiny,
  tabsize=2,
}

%%%%%%%%%%%%%%%%%%%%%%図を通し番号にする%%%%%%%%%%%%%%%%%%%%%%%%%%%%%%%%
%このWebサイトを見てやってみた→ http://rexpit.blog29.fc2.com/blog-entry-99.html
\usepackage{remreset} 
\makeatletter
%	\@removefromreset{figure}{chapter}
%	\def\thefigure{\arabic{figure}}

	\@removefromreset{table}{chapter}
	\def\thetable{\arabic{table}}

%	\@removefromreset{equation}{chapter}
%	\def\theequation{\arabic{equation}}

	\newenvironment{tablehere}
  {\def\@captype{table}}
  {}
\newenvironment{figurehere}
    {\def\@captype{figure}}
    {}

	%\@removefromreset{subfigure}{chapter}
	%\def\thesubfigure{(\alph{subfigure})}
	%\def\p@subfigure{\arabic{figure}}
	%\@removefromreset{subtable}{chapter}
	%\def\thesubtable{(\alph{subtable})}
	%\def\p@subtable{\arabic{table}}
	%最新の TeX では \subfigure, \subtable は廃止されている。(代替コマンド: \subcaption)
%\makeatother
%%%%%%%%%%%%%%%%%%%%%%%%%%%%%%%%%%%%%%%%%%%%%%%%%%%%%%%%%%%%%%%%%%%%

%\usepackage{ascmac}
%\usepackage{mypaper}

%\input{preamble}


\pagestyle{headings}
% \setlength{\topmargin}{1.6cm}
% \setlength{\oddsidemargin}{1.65cm}
% \setlength{\textheight}{20cm}
% \setlength{\textwidth}{32zw} %style
\renewcommand{\baselinestretch}{1.3}
\newlength{\gralen}

%%%%%%%%%%%%%%%%%%%%%%%%丸で囲んだ文字%%%%%%%%%%%%%%%%%%%%%%%%%%%%
%\def\MARU#1{raisebox{.2ex}[0pt]{\leavevmode\setbox0\hbox{$\bigcirc$}%
%\copy0\kerm-\wd0\hbox to\wd0{\hfil{\scriptsize#1}\hfil}}}
\def\MARU#1{{\ooalign{\hfil#1\/\hfil\crcr\raise.167ex\hbox{\mathhexbox20D}}}}%

%%%%% Theorem environment 定理環境 %%%%%%%%%%%%%%%%%%%%%%%%%%%%%%%%%%%%%%%%%%%%
%\newtheorem{ex}{例}[section]
%\newtheorem{df}{定義}[section]
\newtheorem{theo}{定理}[section]
%\newtheorem{lem}{補題}[section]
%\newtheorem{kei}{系}[section]
%\newenvironment{prove}{\par\hspace*{-1.5zw}\begingroup({\bf 証明})}{\par\vspace{1zw}\endgroup}
\newcommand{\vect}[1]{\mbox{\normalsize\boldmath{$#1$}}}

%%%%% boldmath macro %%%%%%%%%%%%%%%%%%%%%%%%%%%%%%%%%%%%%%%%%%%%%%%%%%%%%%%%%%
\def\BM#1{{\mbox{\boldmath $#1$}}}
\def\bm#1{{\mbox{\footnotesize \boldmath $#1$}}}
\newcommand{\seq}[1]{\mbox{\boldmath $#1 $}}

%%%%% □ %%%%%%%%%%%%%%%%%%%%%%%%%%%%%%%%%%%%%%%%%%%%%%%%%%%%%%%%%%%%%%%%%%%%%%
\setlength{\jot}{8pt}
\def\QED{\hfill$\square$}

%%%%% ↑ %%%%%%%%%%%%%%%%%%%%%%%%%%%%%%%%%%%%%%%%%%%%%%%%%%%%%%%%%%%%%%%%%%%%%%
\newcommand{\UpArrow}{\mbox{\kern-.2em$\uparrow$\kern-.05em}}

%%%%% MP DP DMP %%%%%%%%%%%%%%%%%%%%%%%%%%%%%%%%%%%%%%%%%%%%%%%%%%%%%%%%%%%%%%%%%%
\newcommand{\MP}{M\kern-0.17emP}
\newcommand{\DP}{D\kern-0.17emP}
\newcommand{\DMP}{D\kern-0.17emM\kern-0.17emP}
\newcommand{\argmax}{\mathop{\rm argmax}\limits}
\newcommand{\argmin}{\mathop{\rm argmin}\limits} 

%%%%% footnote 脚注 %%%%%%%%%%%%%%%%%%%%%%%%%%%%%%%%%%%%%%%%%%%%%%%%%%%%%%%%%%%
\makeatletter
\footnotesep0pt
\def\@fnsymbol#1{\ifcase#1\or \dagger\or%  %ダガー指定
    \dagger\kern-2pt\dagger\or 
    \dagger\kern-2pt\dagger\kern-2pt\dagger\or
    \dagger\kern-2pt\dagger\kern-2pt\dagger\kern-2pt\dagger\else
    \dagger\kern-2pt\dagger\kern-2pt\dagger\kern-2pt\dagger\kern-2pt\dagger\fi\relax}
\def\@makefnmark{\nolinebreak\hbox{$^{\scriptscriptstyle\@thefnmark}$}}
\def\footnoterule{\kern-5\p@ \hrule width 37mm height0.1pt \kern 4\p@} %\kern 2.6\p@}
\def\thefootnote{\fnsymbol{footnote}}
% Page-Dependent Footnotes by S. Fujita 1995/06/06
\newcount\@fn@total \@fn@total\z@ \newcount\CurrP@ge \CurrP@ge\z@
\def\ftnt@page#1#2{\@tempcnta=#2 \ifnum\@tempcnta>\CurrP@ge
\expandafter\gdef\csname @ftnt@#1\endcsname{#2}\CurrP@ge=\@tempcnta\fi}
\def\Footnote#1{\global\advance\@fn@total\@ne
  \@ifundefined{@ftnt@\the\@fn@total}{\footnote{#1}}
  {\c@footnote\z@ \footnote{#1}}\edef\the@fn@total{\the\@fn@total}%
   {\let\the\z@\edef\next{\write\@auxout{\string\ftnt@page
   {\the@fn@total}{\the\c@page}}}\next}}
\long\def\@makefntext#1{\parindent 1zw\noindent%
  \hbox to 2zw{\hss$^{\@thefnmark}$~}%
  \@tempdima\columnwidth\advance\@tempdima-2zw\parbox[t]{\@tempdima}{#1}}
\makeatother

\renewcommand{\bibname}{参考文献}



%%%%% 本文 %%%%%%%%%%%%%%%%%%%%%%%%%%%%%%%%%%%%%%%%%%%%%%%%%%%%%%%%%%%%%%%%%%%

\begin{document}

\thispagestyle{empty}
\begin{center}
\vspace*{0mm}\hspace*{-0cm}
{\large 卒業論文 (2019年度)}\\
\vspace*{15mm}\hspace*{-0cm}
{\LARGE\bf Knowledge Graph Attention Network }\\
\vspace*{5mm}\hspace*{-0cm}
{\LARGE\bf に基づく顧客分析手法に関する研究}\\
%\vspace*{60mm}\hspace*{-0cm}
%{\Large\bf Performance Analysis of a Decoding Method of}\\
%\vspace*{5mm}\hspace*{-0cm}
%{\Large\bf LDPC Codes over Channels with Memory}\\
\vspace*{60mm}\hspace*{-0cm}
{\Large 指導教員 後藤 正幸}\\
\vspace*{30mm}\hspace*{-0cm}
{\large 早稲田大学創造理工学部}\\
\vspace*{1mm}\hspace*{-0cm}
{\large 経営システム工学科}\\
%\vspace*{5mm}\hspace*{-0cm}
%{\large 情報数理応用研究}\\
\vspace*{10mm}\hspace*{-0cm}
{\Large 1X16C011-1 伊藤 史世}\\
\end{center}


\chapter*{内容梗概}
\pagenumbering{roman}
%%%%%%%%%%%%%%%%%%%%%%%%%
近年ではEC市場規模の拡大に伴い,
各ユーザの購買行動を分析に基づく効果的なマーケティング施策の重要性が高まってきている.
そこで,機械学習の分野においては,購買行動から各ユーザの嗜好を学習し,
ユーザの嗜好に合致したアイテムを推薦するモデルの研究が多く行われている.
ユーザの嗜好を捉える際,購買したアイテムのみでなく,アイテムに関する補助情報を考慮することが必要であると考えられ,
補助情報を取り入れた推薦システムが多く用いられている.
しかし,従来の多くの手法ではその補助情報間の既知の関係性を考慮してモデルが構築されていないため,
アイテムと補助情報間の高次の関係性を捉えることが困難である.
このような中,Deep Learning 手法の1つとしてユーザ,アイテムおよびその補助情報の高次の関係性をグラフ状のデータ構造で表す手法
Knowledge Graph Attention Network(以下,KGAT)が存在する.
このモデルではグラフを用いて,個別のユーザに対するアイテムの推薦理由について解釈可能であり,
従来のモデルの問題点であったデータ間の高次の関係性の表現に加えて,Deep Learing を用いる問題点であった
解釈性を解決している.

しかし,マーケティングの分野では,
ユーザの嗜好は同一の属性をもつユーザである程度類似しているという前提に基づき,
ユーザを属性情報を用いてセグメンテーションし,セグメントごとに施策を実施する(以下,マスマーケティング)ことが多い.
そのため,ユーザ属性を用いたセグメントに対する解釈が求められており,この問題を解決する手法の提案が望まれる.

そこで本研究では,ユーザの嗜好が,「ユーザの属性により説明される嗜好」と「ユーザ固有の嗜好」から構成されると仮定し,
各ユーザの嗜好における属性情報の影響度を定量化する手法について研究を行い,KGATを用いてユーザの嗜好をユーザの属性情報と購買アイテムからモデル化する手法を提案した.
この手法を用いることにより,属性情報の影響度が高いユーザに対してはマスマーケティングを実施し,
影響度が低いユーザに対しては,ユーザにパーソナライズした施策を実施するなど,
より効果的なマーケティング戦略を取ることが可能になると考えられる.
最後に,提案手法の有用性について検討するために,実際のEC市場の評価値データを分析に適用し,得られる結果について考察を与えた.

\tableofcontents


\chapter{序論}
\pagenumbering{arabic}

\section{研究背景}
近年EC市場規模の拡大に伴い,ユーザの購買行動に関する膨大なデータが蓄積されている.
この蓄積されたデータからユーザの購買行動を分析し,効果的なマーケティング施策を実施する重要性が高まっている.
そこで,機械学習の分野においては,購買行動から各ユーザの嗜好を学習し,
ユーザの嗜好に合致したアイテムを推薦するモデルの研究が多く行われている.
このように,ユーザ個人に対して異なる施策を実施する方法をマーティングの分野においては,One to One マーケティングといい,情報通信技術の発展に伴い注目を集めている.
最適なOne to One マーケティングを実施するためには,ユーザ個人の嗜好を正確に捉える必要がある.
ユーザの嗜好を正確に捉えるためには,購買したアイテムのみでなく,アイテムに関する補助情報を考慮することが必要であると考えられ,現在では補助情報を取り入れた推薦システムが多く用いられている\cite{rendle2010factorization}.
しかし,従来の多くの手法ではその補助情報間の既知の関係性を考慮してモデルが構築されていないため,
アイテムと補助情報間の高次の関係性を捉えることが困難であり,複雑な関係性をデータからうまく学習することができない.
さらに最近では,Deep Learning を用いた推薦システムがその精度とモデルの柔軟性から評価され,多くの研究が行われている\cite{2019DeepRecSurvey}.しかし,既存のDeep Learning 手法の多くは,モデルに対する解釈を与えることが困難であり,モデルの学習によって得られた結果をアイテムの推薦以外に活かすことができないという問題点があった.
このような中,Deep Learning 手法の1つとしてユーザ,アイテムおよびその補助情報の高次の関係性をグラフ状のデータ構造で表す手法
Knowledge Graph Attention Network(以下,KGAT)\cite{KGAT19}が存在する.
このモデルでは,補助情報を含んだデータからなるグラフをモデルの入力とし,
既知の関係性を捉えることで,最新のDeep Learning を用いた手法からその精度を向上させている.
さらにグラフを用いて,個別のユーザに対するアイテムの推薦理由について解釈可能であり,
従来のモデルの問題点であったデータ間の高次の関係性の表現に加えて,Deep Learning を用いる問題点であった解釈性を与えることに成功している.
このように,より良いOne to One マーケティングを実現するための推薦システムに関する研究が盛んに行われており,様々なモデルが提案されている.

しかし,実際の企業では依然として,ユーザを属性情報を用いてセグメンテーションし,セグメントごとに施策を実施する(以下,セグメントマーケティング)ことが多い.セグメントマーケティング では,One to One マーケティング と異なり多くのユーザを対象とするため,企業のブランディングに繋がり,現在でも重要視されている.
しかし,セグメントマーケティングでは,ユーザの嗜好は同一の属性をもつユーザである程度類似しているという前提に基づいている.
ユーザの嗜好が多様化している昨今では,同一セグメント内でもその嗜好にはバラつきが生じており,セグメント毎に画一的な施策を実施することではニーズに応えることができないユーザが存在することが考えられる.
さらに,近年ではユーザが興味がないと感じたネット上の広告を表示しないようにすることが可能で,画一的な施策を行う際の懸念事項となっている.
そのため,ユーザの嗜好が多様化している現在ではOne to One マーケティングとセグメントマーケティングを組み合わせることにより,効果的なマーケティングを行うことが必要であること考えられ,これを実現するための手法の提案が望まれる.


\section{研究目的}
本研究では,より効果的なマーケティング施策の立案を可能にするために,ユーザの嗜好が,「ユーザの属性により説明される嗜好」と「ユーザ固有の嗜好」から構成されると仮定し,
各ユーザの嗜好における属性情報の影響度を定量化することを考える.
これにより,属性情報の影響度が高いユーザに対してはセグメントマーケティングを実施し,
影響度が低いユーザに対しては,One to One マーケティングを実施するといった戦略を取ることが可能になると考えられる.
% TODO:なぜこの戦略が効果的であるか述べる
% 適切な人にブランディングを実行し,そうでない人には不要な広告を表示させなくていい
具体的には,ユーザとその属性情報,アイテムとその補助情報からなるグラフに対してKGATを適用することで,ユーザの嗜好をユーザの属性情報と購買アイテムからモデル化する手法を提案する.これにより,ユーザの嗜好を属性と購買アイテムの双方の観点からの分析が可能となる.
最後に,提案手法を用いて,実際のECサイトの購買履歴データを分析し,
提案モデルの有用性を示す.


\section{論文の構成}
本論文は,7章から構成されており,以下にその概要を示す.\par
第2章では,本研究で研究対象とするECサイトの概要や特徴,
対象とする問題について述べる.
\par
第3章では,近年の推薦システムの研究動向について述べる.
\par
第4章では,従来手法の研究背景やその概要について述べる.
\par
第5章では,提案モデルの概要,および構築手順について述べる.
\par
第6章では,提案モデルを用いたECサイトの評価値データを用いた分析の概要と結果,および考察を述べる.
\par
第7章では,本研究全体についての考察を述べる.
\par
第8章では,本研究のまとめと今後の課題を述べる.


 % 概要
\chapter{準備}
本章では,まず一般的な自動分類問題について説明した後,本研究で扱う半教師あり学習について述べる.さらに本研究で用いる二値判別器を説明し,二値分類問題を多値判別問題に拡張する手法であるECOC法について述べる.

%%%%%%%%%%%%%%%%%%%%%%%%%%%%%%%
\section{自動分類問題}
%%%%%%%%%%%%%%%%%%%%%%%%%%%%%%%
%\label{sec:zirei}
自動分類問題\cite{bis1},\;\cite{bis2}とは,カテゴリ情報があらかじめ与えられた学習データから分類規則を学習し,カテゴリが未知の入力データのカテゴリを推定する問題である \cite{Nishida08}.
ここで,${\bm x} \in \mathbb{R}^{d}$を$d$次元特徴ベクトル,$\mathcal C = \{c_1,...,c_k,...,c_K\}$をカテゴリラベルの集合,$y\in \mathcal C$とすると,$N$個の学習データ集合$\mathcal D = \{{\bm x}_{n},y_{n}\}^{N}_{n=1}$をもとに${\bm x}$のカテゴリラベルを$K$個のカテゴリ$\mathcal C$の中から1つに決定する問題である.
すなわち,これを達成するための関数$f:{\bm x}→y$を学習するための問題であるといえる.また,以降で扱う二値分類問題では,$K = 2$でありカテゴリ集合$y$は$\{+1,-1\}$とする.

データの分類は一般的に以下の処理を行うことにより実現される.

(1)特徴抽出

(2)分類規則の学習

(3)データの分類
% 分類問題文献
%\bibitem{Nishida08}
%K. Nishida, \lq \lq Learning and detecting concept drift,'' \textit{researchgate.net}, 2008

%そして,「特徴の抽出」と「分類器の学習」,「分類器による予測」の3段階に分かれる.
%

特徴抽出とは,分類に用いる各データの特徴量を構成するための処理である.すなわち,機械による自動処理が可能になるようにデータの特徴量を数値化しベクトルに変換することで,特徴ベクトルとして表現することである.

分類規則の学習とは,入力データの特徴ベクトルとそのカテゴリをセットにした学習データを用いて,新規入力データがどのカテゴリに所属するかを決定するための規則を学習することである.分類規則は,同一カテゴリに属するデータの特徴を捉え学習される.

データの分類とは,カテゴリが未知の新規入力データを,学習をして得た分類規則に従い一つのカテゴリに分類することである.
%カテゴリが未知の新規入力データ${\bm x}_{new}$の所属カテゴリを推定する.
\par


%%%%%%%%%%%%%%%%%%%%%%%%%%%%%%%
\section{半教師あり学習}
%%%%%%%%%%%%%%%%%%%%%%%%%%%%%%%
%\label{sec:taisyoudata}
一般的に分類問題を扱う際に行われる学習は,教師あり学習と呼ばれるものである.教師あり学習では全ての学習データにおいて,特徴ベクトルとそれに対応するカテゴリラベルが与えられている.例えば,動物の写った画像データが与えられたとき,全ての各画像に対して,{イヌ,ネコ,ゾウ,...}といったカテゴリの内のいずれかの適切なカテゴリが対応した状態である.\par
教師あり学習は分類問題を扱う際,学習データが多いとき良い分類境界の学習を行うことができる.しかしながら実世界に存在する多くのデータにはカテゴリが付与されていない.大量のデータにラベルを付与する作業は,時間,費用の二つの面において大きな問題点となっている.一方,少数のラベルありデータのみを学習した場合,データ数が少ないためそのデータに依存しやすく過学習してしまう.\par
そこでこれらの問題点を解消する手法として半教師あり学習が近年注目されている.半教師あり学習では,少数のラベルありデータと多数のラベルなしデータを用いることで分類規則を構築する手法である.この手法は,多数のラベルなしデータを用いることで本来望まれる分類境界に近づけることを目的とした手法である.


\vspace{2.0mm}
\begin{figurehere}
\begin{center}
  \includegraphics[width=10.0cm]{hankyoshi.png}\par
\caption{機械学習の種類\cite{semi}}
\label{機械学習の種類}
\end{center}\par
\end{figurehere}
%\vspace{1.0mm}
\newpage
%本論文では,特徴ベクトル$\bm{x}\in\mathbb{R}^d$のカテゴリラベルを$K$個のカテゴリ$\mathcal C =\{c_1,...,c_k,...,c_K\}$の中から1つに決定する多値分類問題を扱う.
半教師あり学習では,ラベルありデータとラベルなしデータを扱うためそれぞれ別のデータ集合として定義する.ラベルありデータ集合は$\mathcal D_L = \{\bm{x}_i^L,y_i^L\}_{i=1}^{N_L}$と定義し,特徴ベクトル$\bm{x}_i^L$とそれに対応したカテゴリラベル$y_i^L$が与えられている.ラベルなしデータ集合を$\mathcal D_U = \{\bm{x}_i^U,*\}_{i=1}^{N_U}$と記述し,$*$は未観測であることを表す.すなわち,特徴ベクトル$\bm{x}_i^U$に対するカテゴリラベルが与えられていない.ここで,$\bm{x}_i^L\in\mathbb{R}^d$,$\bm{x}_i^U\in\mathbb{R}^d$,ラベルありデータ数を$N_L$,ラベルなしデータ数を$N_U$とする.また,半教師あり学習は少数のラベルありデータと多数のラベルなしデータを用いるため,一般に$N_L\ll N_U$を前提とする.



%少数のラベルありデータ集合$\mathcal D_L = \{\bm{x}_i^L,y_i^L\}_{i=1}^{N_L}$と多数のラベルなしデータ集合$\mathcal D_U = \{\bm{x}_i^U,*\}_{i=1}^{N_U}$を用いて学習を行う.ここで,$\bm{x}_i^L\in\mathbb{R}^d$,$\bm{x}_i^U\in\mathbb{R}^d$とする.ラベルありデータ数を$N_L$,ラベルなしデータ数を$N_U$とし,一般に$N_L\ll N_U$を前提とする.

ラベルありデータ集合は$\mathcal D_L$を用いてラベルなしデータ$\bm{x}_i^U$のラベル$y_i^U$を推定し,仮のラベル(以下,仮ラベル)を付与することで,これら全てのデータを用いて分類器の学習を行う.
%カテゴリが未知の入力データに対し,所属カテゴリを推定する.
%$\tilde{\vector{x}}$$\tilde{y}\in \mathcal C$

%%%%%%%%%%%%%%%%%%%%%%%%%%%%%%%
\section{二値判別器}
%%%%%%%%%%%%%%%%%%%%%%%%%%%%%%%
本節では,本研究において用いる学習器について述べる.
\subsection{Support Vector Machine}
Support Vector Machine(以下,SVM) \cite{VV}は1995年にV.Vapnikにより考案された二値分類手法で,強力な汎化能力を持った手法として現在でも広く応用されている.SVMは,クラス間マージンの最大化を行うことで分離超平面を得る識別関数の学習法である \cite{Bryant99}.マージンの最大化は,学習データにより与えられた不等式制約条件下で最適化問題を解くことにより得られる.

学習データ${\bm x}_{n}\in \mathbb{R}^{d}$の所属するカテゴリが正例ならば$y_{n}=+1$,負例ならば$y_{n}=-1$であるとき,各カテゴリの学習データが線形分離可能であると仮定すると,各データ${\bm x}_{n}$に対して,次の式が成り立つ.

\begin{eqnarray}
f({\bm x}_{n})={\bm w}^{\mathrm{T}}{\bm x}_{n}+b
\end{eqnarray}
\begin{eqnarray}
y_{n}=
\begin{cases}
+1 &f({\bm x}_{n})\ge \kappa \\
-1 & f({\bm x}_{n})< \kappa 
\end{cases}
\end{eqnarray}
%y_{n}=
ここで,$f(\cdot)$は分類器,${\bm w}=(w_{1},w_{2},...,w_{d})^{\mathrm{T}}$は係数ベクトル,$b$はバイアス項,$\kappa$はマージンを表す.ここでマージンとは,分類器から得られる識別超平面とその超平面に対して最近傍の学習データ(サポートベクトル)との距離のことを示す.
%,${\bm w}^{T}{\bm x}_{n}$は係数ベクトルと各学習データの内積
式(2.2)をマージン$\kappa$で正規化し,正規化した係数ベクトルとバイアス項をそれぞれ${\bm w}$と$b$で再定義すると$f({\bm x}_{n})$は式(2.3)と表すことができる.

\begin{eqnarray}
f({\bm x}_{n})={\bm w}^{\mathrm{T}}{\bm x}_{n}+b
\end{eqnarray}
\begin{eqnarray}
y_{n}=
\begin{cases}
+1 &f({\bm x}_{n})\ge \kappa \\
-1 & f({\bm x}_{n})< \kappa 
\end{cases}
\end{eqnarray}

この式は,
\begin{eqnarray}
y_{n}({\bm w}^{\mathrm{T}}{\bm x}_{n}+b)-1\ge0 ,\;\;n=1,...,N
\end{eqnarray}
と表すことができる.

式(2.3)を満たす超平面は無数に存在する.このとき,最大マージンを持つ分類器から構築される超平面を最適分離超平面と呼び,以下に示す最適化問題を解くことにより,最適分離超平面を構築する分類器を求める.\par
%\newpage
{\bf 最適化問題}
\begin{eqnarray}
\text{minimize} & \frac{1}{2} {\bm w}^{\mathrm{T}}{\bm w} \nonumber \\
\text{subject\;\;to} & y_{n}({\bm w}^{\mathrm{T}}{\bm x}_{n}+b)-1\le0
\end{eqnarray}

式(2.4)は制約付き最適化問題であるので,ラグランジュ未定乗数法 \cite{Platt98}を用いて解くことができる.各不等式の制約に対応する未定乗数${\bm a}=(a_{1},a_{2},...,a_{N})^{\mathrm{T}}\;(a_{n}\le 1)$を用いてラグランジュ関数を
\begin{eqnarray}
L({\bm w},{\bm a})=\frac{1}{2}||{\bm w}||^{2}+\sum^{N}_{n=1}a_{n}(1-y_{n}(w_{0}+{\bm w}^{\mathrm{T}}{\bm x}_{n}))
\end{eqnarray}
と定式化することができる.

これを用いて,以下の式(2.8)で定義される双対問題を解くことにより最適化を得る.
\begin{eqnarray}
\text{maximize} & \sum^{N}_{n=1}\alpha_n - \frac{1}{2}\sum^{N}_{n=1}\sum^{N}_{m=1}\alpha_{n}\alpha_{m}y_{n}y_{m}{\bm x}_{n}^{T}{\bm x}_{m}\nonumber \\
\text{subject\;\;to} & \sum^{N}_{n=1}\alpha_{n}t_{n}=0
\end{eqnarray}


以下にSVMの分類境界を表す例をあげる.
\begin{figure}[H]
\centering
\includegraphics[width=10.0cm]{svm.png}
\caption{SVMの分類境界}
\label{SVM}
\end{figure}

\subsection{決定木}
決定木\cite{Safavian02},-,\cite{Wilkinson92}とは,目的変数と説明変数間の関係を木構造として捉える機械学習の一手法である.決定木は根ノードから葉ノードへ向かう途中の中間ノードで簡単な分類規則を割り当て,最終的に葉ノードに割り当てられたカテゴリを分類先とする分類器であり,単純な分類規則を複数組み合わせることで複雑な分類規則を得ることができる手法である.また,この手法は中間ノードの過程を見ることで分類に有効な特徴量や値を解析する際にも有効であると知られている.

説明変数$\bm{x}=(x_{1},x_{2})^T$が与えられたもとで予測対象の目的変数$y \in \{-1,+1\}$を予測する場合$x_1$の閾値$\theta_1$と$x_2$の閾値$\theta_2$をもとに,図\ref{kyoukaitree}に示す分類境界が得られる.

\begin{figure}[H]
\centering
\includegraphics[width=5.0cm]{kyoukaiki.png}
\caption{決定木における識別境界の例}
\label{kyoukaitree}
\end{figure}
上記の例の場合,図\ref{tree}に示すような決定木として表現できる.
\begin{figure}[H]
\centering
\includegraphics[width=6.0cm]{tree.png}
\caption{決定木}
\label{tree}
\end{figure}


%%%%%%%%%%%%%%%%%%%%%%%%%%%%%%%%%%%%%%%%%%%%%%%%%%%%%%%%%%%%%%%%%%%%%%%%%%%%%%%%%%%%%%%%%%
\newpage
\section{アンサンブル学習}

複数の分類器を組み合わせることにより全体としての汎化性能の向上を図る方法としてアンサンブル学習
%\footnote{アンサンブル学習は任意の分類器によるアンサンブル手法であるが,本研究では決定木を分類器として使用するものとする.} 
\cite{Liu99},\;\cite{Ueda02},\;\cite{Ueda05}が提案されており,その有効性が示されている.アンサンブル手法の中でも,Bagging \cite{Breiman96},\;\cite{RF},Boosting \cite{Cortes94},\;\cite{Freund97},\;\cite{Hirai12}という手法が多く用いられることが多い.以下にその概要を示す.
\subsection{Bagging}
%\begin{description}
%\item[Bagging]\mbox{}\\
\;\;\;Baggingとは,$B$個の分類器$g_{1},g_{2},...,g_{B}$ごとに学習データのブートストラップサンプル$\mathcal{D}'=\{{D}_{1},{D}_{2},...,{D}_{B}\}$を作成し,それぞれを用いて分類器$g_{1},g_{2},...,g_{B}$を独立に学習する.ここで,ブーストストラップサンプルとは,データ集合$\mathcal{D}$から重複を許し,復元抽出を行うことで学習データのサンプル$({D}_{1},{D}_{2},...,{D}_{B})$を生成する手法である \cite{Miteni99}.新規入力データ$\tilde{\bm{x}}$に対しては,それぞれの分類器出力$g_{1}(\bm{x}),g_{2}(\bm{x}),...,g_{B}(\bm{x})$の式(2.9)の多数決により推定を行う.
\begin{eqnarray}
\hat{y}={\rm sign}(g_{1}(\tilde{\bm{x}})+g_{2}(\tilde{\bm{x}})+,...,+g_{B}(\tilde{\bm{x}}))
\end{eqnarray}

ここで,{\text sign}($x$)は,$x\ge0$のとき+1,$x<0$のとき$-1$をとる関数である.以下の図\ref{ensenble}にBaggingのイメージ図を示す.


\begin{figure}[H]
\centering
\includegraphics[width=9.0cm]{bag.png}
\caption{Baggingの学習イメージ}
\label{ensenble}
\end{figure}

\newpage
\subsection{Boosting}
%\item[]\mbox{}\\ 
\;\;\;Boostingとは,以前までに構築された分類器で誤分類されたデータを正しく分類することができるように新たな分類器を構築することで,全体として分類性能を向上させる手法である.新規入力データ$\tilde{\bm{x}}$に対しては,それぞれの分類器の重み付き多数決により推定を行う.



%以下に,Boostingの代表的な手法であるAdaboostの学習法を示す.


Boostingでは,$T$個の分類器$g_{1},g_{2},...,g_{T}$を用いて,1つ前の学習器の分類結果に基づき誤分類されたとされるデータに着目し,その誤りを正しく分類できるような分類器を順に作成する.

Boostingでは,$T$個の分類器$g_{1},g_{2},...,g_{T}$と重み$\alpha_{1},\alpha_{2},...,\alpha_{T}$により$\tilde{\bm{x}}$の予測カテゴリ$\hat{y}$を
\begin{equation}
\hat{y}={\rm sign}\left(\sum^{T}_{t=1}\alpha_{t}g_{t}(\tilde{\bm{x}})\right)
\end{equation}
として求める.分類器$g_{t}({\bm x})$は,$g_{t-1}({\bm x})$で誤分類したデータに重みを付与することで,誤分類されたデータに焦点を当てた学習が行われ,これを繰り返すことで予測誤差の最小化を目指す.ここで,$t$回目の試行におけるデータに対する重みは${\bm w}_{t}=(w^{t}_{1},w^{t}_{2},...,w^{t}_{N})$となる.また,分類器$g_{t}$に対する重み$\alpha_{t}$は,分類器の誤り率が低いほど大きくなるように決定される.

図\ref{ensenble}にBoostingのイメージを示す.

\begin{figure}[H]
\centering
\includegraphics[width=8.0cm]{boost.png}
\caption{Boostingの学習イメージ}
\label{ensenble}
\end{figure}


%\end{description}

%\newpage
%%%%%%%%%%%%%%%%%%%%%%%%%%%%%%%
\section{ECOC法\cite{ECOC}}
%%%%%%%%%%%%%%%%%%%%%%%%%%%%%%%
複数の二値判別器を組み合わせ,多値判別器を構成する手法としてECOC法(Error-Correcting Output Cording多値判別法)が提案されている.以下では本研究で用いる手法とその復号法について述べる.

\begin{description}
\item[1-versus-the rest 多値判別手法]\mbox{}\\ 
%\subsection{1-vs-the rest 多値判別手法}
\;\;\;二値判別器を複数用いて,多値判別を行う方法の代表的手法\cite{Code}として,1-versus-the rest 多値判別手法が挙げられる.\par
1-versus-the rest 多値判別手法では,全ての$i = 1,2,...,K$に対して判別対象カテゴリ$c_i$とそれ以外のカテゴリに分ける1-versus-the rest 多値判別器を作る.\par
以下で,本手法で利用する符号表について述べる.

\begin{figure}[H]
\centering
表2.1:1-versus-the rest 多値判別法の符号表例
\includegraphics[width=10.0cm]{1vs.png}
%\label{1vs}
\end{figure}\par
%\newpage
表2.1は,カテゴリ数$K = 4$の場合における1-versus-the rest 多値判別法の符号表の例であり,各二値分類器の構成を符号表と呼ばれる$\{+1,-1\}$の二値で表される数値表により表現する.いま符号表を$\bm{W}$とすると,二値分類器の個数はカテゴリ数と同じ$K$個であるため,$\bm{W}$は$K\times K$行列となる.符号表$\bm{W}$の各列ベクトルは二値分類器の構成を表現しており,要素が$+1$のカテゴリ集合と要素が$-1$のカテゴリ集合を二値分類すると解釈できる.例えば,表2.1の1番目の列の判別器では$c_1$と$c_2,c_3,c_4$のカテゴリを分けることを意味する.また,符号表$\bm{W}$の$k$行目の行ベクトルをカテゴリ$c_k$の符号語と呼び$\bm{W}_{c_k}$と表現する.
以下では,本研究で使用するハミング距離を用いた復号法を説明する.



\item[ECOC法に基づく復号法]\mbox{}\\ 
%\subsection{1-vs-the rest 多値判別手法}
\;\;\;二値判別器を複数用いて多値判別を行うが,そのときの復号方法について述べる.また,二値判別器の出力が確率値として出力される場合,分類結果のカテゴリとして表される場合があるためそれぞれについて説明する.


新規入力データ$\tilde{\bm{x}}$に対する$r (1\leq r \leq R)$番目の二値分類器の出力を$g_r(\tilde{\bm{x}})\in\{+1,-1\}$,符号語$\bm{W}_{c_k}$の$r$番目の値を$W_{c_k\cdot r}$と定義する.このとき$\bm{x}$は,符号語$\bm{W}_{c_k}$と分類器の出力$\bm{g}=(g_1(\bm{x}),...,g_R(\bm{x}))$のハミング距離を求め,最も近いカテゴリに分類できる.\par
ハミング距離$d_H$は式(\ref{eq:eq20})により求められる.ここで$\delta(a,b)$は$a = b$のとき$+1$,それ以外は$0$となるインジケータ関数である.

\begin{equation}
\label{eq:eq20}
 \hspace{20pt}d_H(\bm{W}_{c_k},\bm{g}) =\sum_{r=1}^R{\delta(W_{c_k\cdot r} , g_r(\tilde{\bm{x}}))}
\end{equation}\par

%\newpage

また,二値分類器の出力が確率値\cite{rvm} で得られるとし,$r$番目の分類器の出力を$G_r$としたとき,$\bm{x}$の推定所属カテゴリ$\hat{c}_k$は式(\ref{eq:eq1})により求めることができる.

\begin{equation}
\label{eq:eq1}
 \hspace{20pt} \hat{c}_k = \underset{{c_k}}{\text{argmin}} \prod_{r=1}^R G_r^{W_{c_k\cdot r}} (1 - G_r)^{W_{c_k\cdot r}}
\end{equation}\par

これら二つの復号方法があり,確率値として出力されるものについては,一般的にどちらの復号方法を行っても分類精度は同程度になると言われている.また,ハミング距離の欠点として,最小距離復号を行う際に複数のカテゴリで等距離になることが多いため,カテゴリを正しく推定できないことがある.

\end{description}



\newpage
















\chapter{関連研究}

本章では,推薦システムの概要を述べると共に,近年の研究動向,および代表的な手法の概要について説明する.

%%%%%%%%%%%%%%%%%%%%%%%%%%%%%%%%%%%%%%%%%
\section{推薦システムの概要}
%%%%%%%%%%%%%%%%%%%%%%%%%%%%%%%%%%%%%%%%%
\label{sec:Rec_intro}

推薦システムとは,データからユーザの嗜好の合致したアイテムなどを選び出し,ユーザに提示するシステムであり,大きく以下の2つに大別される.




\newpage
SemiBoostはデータ間の類似度を利用し,ラベルなしデータに仮ラベルを付与することで分類器を学習する.ここで作成した分類器を用いて分類を行い,分類により付与された仮ラベルと他のデータとの類似度を測ることで正しく分類できているか評価する.\par
その後,付与されたラベルが誤っている可能性の高いデータが正しく分類されるように,誤分類している可能性の高いラベルなしデータを抽出してそのデータに着目した弱分類器を構築する.弱分類器を以前の分類器に重みをかけて付加することで新たな分類器を構築する操作を繰り返す.\par

このように新たな分類器を逐次的に構築し,構築されたすべての分類器の結果をアンサンブルすることにより分類器を構築する.半教師あり学習で用いる少数のラベルありデータは,何らかの行動の結果として得られているデータであり,実際のデータ分布に従ったランダムサンプリングではない場合があるが,SemiBoostは二つのデータの分布が明瞭に分かれているときに,少数のラベルありデータによる半教師あり学習の効果が大きい傾向が見られる.データの分布が同一カテゴリ内でも複数に分かれているときは,精度の改善が得られにくいことが問題として挙げられる.
%\newpage
%%%%%%%%%%%%%%%%%%%%%%%%%%%%%%%%%%%%%%%%%
\section{SemiBoostアルゴリズム}
%%%%%%%%%%%%%%%%%%%%%%%%%%%%%%%%%%%%%%%%%
\label{sec:SemiBoost}
以下にSemiBoostのアルゴリズムを示す.


\begin{description}
  \item[Step1)] 全学習データ間の類似度を式(3.4)によるガウスカーネルを用いて求める.

  \item[Step2)] アンサンブル分類器$H(\bm{x})$を初期化し,$t = 1$とする.

  \item[Step3)] 全てのラベルなしデータ$\bm{x}_i^U$の仮ラベル$y_i^U$を式(\ref{eq:eq6})を用いて予測する.

  \item[Step4)] 誤分類されている可能性の高いラベルなしデータ$\bm{x}_i^U$を式(3.9)により学習に用いるデータとして選択する.

  \item[Step5)] Step4で選んだラベルなしデータ$\bm{x}_i^U$を用いて弱分類器$h_t$を学習し,分類誤差$\varepsilon_t$を用いて分類器の重みを決定する.
ここで$h_{t,i}$はラベルなしデータ$\bm{x}_i^U$を入力としたときの弱分類器$h_t$による二値分類結果である.
%\begin{equation}
\begin{eqnarray}
%\label{eq:eq8}
 %\hspace{30pt}
 \alpha_t &=& \frac{1}{4}\ln\left({\frac{1-\varepsilon_t}{\varepsilon_t}}\right)\\
%\end{equation}\par
%\begin{equation}
%\label{eq:eq9}
 %\hspace{-10pt}
\varepsilon_t &=& \frac{\sum_{i=1}^{N_U}p_i\delta(h_{t,i} ,-1)+\sum_{i=1}^{N_U}q_i\delta(h_{t,i} ,1)}{\sum_{i}(p_i+q_i)}
%\end{equation}\par
\end{eqnarray}

  \item[Step6)] Step5で学習した弱分類器$h_t$を$\alpha_t$で重みづけし,アンサンブル分類器$H(\bm{x})$に結合し更新する.
\begin{equation}
\label{eq:eq10}
 \hspace{20pt}H(\bm{x})\leftarrow H(\bm{x})+\alpha_t h_t(\bm{x})
\end{equation}\par

  \item[Step7)] $t \leftarrow t+1$とし,Step4へ.$t = T$のとき終了.

\end{description}\par
\vspace{2.0cm}


Step1で用いるガウスカーネル\cite{kernel}は式(\ref{eq:eq4})で表される.$s_{ij}$は$\bm{x}_i$と$\bm{x}_j$の類似度であり,$\bm{x}_i,\bm{x}_j$は全データのうち2つの任意の点とする.
ただし,$\|\cdot \|_2^2$は$\ell_2$ノルムの2乗を表し,$\sigma$は定数とする.
\begin{equation}
\label{eq:eq4}
 \hspace{30pt} s_{ij} = \exp\left(-\frac{\|\bm{x}_i -\bm{x}_j \|_2^2}{\sigma^2}\right)
\end{equation}\par
%この$s_{ij}$をまとめた全データ間の類似度行列を作成する.
%ラベルありデータとラベルなしデータの類似度を$s_{ij}^{UL}$,ラベルなしデータ同士の類似度を$s_{ij}^{UU}$として以降のステップで用いる.
\par


Step2のアンサンブル分類器$H(\bm{x})$は,全ての弱分類器の出力結果をアンサンブルした結果を出力する分類器である.このアンサンブル分類器に弱分類器を付加して分類器を構築する.式(\ref{eq:eq24})によってアンサンブル分類器を初期化する.
\begin{equation}
\label{eq:eq24}
 \hspace{30pt} H(\bm{x}) = 0
\end{equation}\par
$H(\bm{x}) = 0$とは,入力値をそのまま返す状態である.例えば,初期状態のラベルが$1$であれば,そのまま$1$を出力する状態である.\par

\newpage
Step3において,ラベルなしデータ$\bm{x}_i^U$がカテゴリ$+1$に所属する可能性が高いときに大きな値をとる$p_i$と,カテゴリ$-1$に所属する可能性が高いときに大きな値をとる$q_i$を用いて,ラベルなしデータがどちらのラベルに所属するかを式(\ref{eq:eq6})により推定する.$p_i, q_i$の式は,前項でラベルありデータとの類似度の評価,後項で他のラベルなしデータとの類似度の評価を行うことで,ラベルなしデータ$\bm{x}_i$が誤分類されていないかの評価を行っている.
\begin{eqnarray}
\label{eq:eq6}
 \hspace{40pt} y_i^U = \rm{sign}\it{(p_i - q_i)}
\end{eqnarray}\par
ただし,$\rm{sign}(\it a)$は,$a \geq 0$のとき$+1$,$a < 0$のとき$-1$をとる関数である.\par

また,このとき用いる$p_i, q_i$は式(\ref{eq:eq2}),(\ref{eq:eq3})により求める.
\begin{equation}
\label{eq:eq2}
 \hspace{-12pt} p_i =\sum_{j=1}^{N_L} s_{ij}^{UL} e^{-2H_i} \delta(y_j^L ,1) +\frac{\gamma}{2} \sum_{j=1}^{N_U} s_{ij}^{UU} e^{H_j-H_i}
\end{equation}\par
\begin{equation}
\label{eq:eq3}
 \hspace{-12pt} q_i =\sum_{j=1}^{N_L} s_{ij}^{UL} e^{2H_i} \delta(y_j^L ,-1) +\frac{\gamma}{2} \sum_{j=1}^{N_U} s_{ij}^{UU} e^{H_i-H_j}
\end{equation}\par
$\delta(a,b)$は$a = b$のとき$+1$,それ以外は$0$となるインジケータ関数であり,$\gamma  (> 0)$は定数である.$s_{ij}^{UL} (\geq 0)$はラベルなしデータ$\bm{x}_i^U$とラベルありデータ$\bm{x}_j^L$の類似度,$s_{ij}^{UU}(\geq 0)$はラベルなしデータ$\bm{x}_i^U$と他のラベルなしデータ$\bm{x}_j^U$の類似度とする.$H_i$は,アンサンブル分類器に$\bm{x}_i^U$を入力したときの出力値である.\par
%仮ラベルを付与して

Step4において,ラベルなしデータ$\bm{x}_i^U$の中から弱分類器の学習に用いるデータを式(3.9)の確率に従い選択する.
ここで,$P(\bm{x}_i^U)$はラベルなしデータ$\bm{x}_i^U$が学習に用いられる確率である.
\begin{equation}
\label{eq:eq7}
 \hspace{20pt} P(\bm{x}_i^U) = \frac{|p_i-q_i|}{\sum_{i=1}^{n_U}|p_i-q_i|}
\end{equation}\par
また,学習に用いるデータの数は任意に決定することができるが,経験的にラベルなしデータのうち$10\%$程度が好ましいことが知られている.

\vspace{1.0cm}

Step5において,Step4で選択したラベルなしデータとラベルありデータを用いて弱分類器の学習を行う.このとき,弱分類器として決定木やSVMなどの二値分類器などが用いられる.また,この弱分類器の重みは誤分類したものに大きな重みを付与するBoostingの考え方を用いて導出されている.


\vspace{1.0cm}
%%%%%%%%%%%%%%%%%%%%%%%%%%%%%%%%%%%%%%%%%
%\begin{flushleft}
%\large{\bfseries {(3) トレンド関数}}
%\end{flushleft}
%%%%%%%%%%%%%%%%%%%%%%%%%%%%%%%%%%%%%%%%%
SemiBoostのイメージを図\ref{semi2},図\ref{semi3}に示す.
\vspace{1.0cm}
\begin{figure}[H]
\centering
\includegraphics[width=12.0cm]{semiboost1.png}
\caption{アンサンブル分類器の更新前}
\label{semi2}
\end{figure}
\vspace{1.0cm}
\begin{figure}[H]
\centering
\includegraphics[width=12.0cm]{semiboost2.png}
\caption{アンサンブル分類器の更新後}
\label{semi3}
\end{figure}




\newpage
%%%%%%%%%%%%%%%%%%%%%%%%%%%%%%%%%%%%%%%%%
\section{SemiBoostの課題}
%%%%%%%%%%%%%%%%%%%%%%%%%%%%%%%%%%%%%%%%%
\label{sec:SemiBoost_prob}
SemiBoostは,データの類似度をもとにした半教師あり学習の手法であるため分類精度の高い手法であるが,以下の問題点が挙げられる.\par
\vspace{3.0mm}
\begin{description}
\item[問題点]\mbox{}\\
\end{description}
\vspace{-10.0mm}

\begin{itemize}
\item 一般的な分類問題は多値分類問題の場合が多いが,この手法は二値分類問題を想定した問題なので多値分類問題に直接適用することができない.\par
\item データの分布が複雑なときは仮ラベルの付与が難しくなり,良いアンサンブル分類器の構築が難しい.\par
\item データの類似度を用いて学習を行うため,学習時間が必要.

\end{itemize}

本研究では,この問題のうち上二つの問題を解決することを目的としている.









 
\chapter{提案手法}

本章では,本研究における提案モデルについて述べる.
\ref{sec:提案への着想}節では,提案モデルへの着想について述べる.
\ref{sec:ラベル付け}節では,ラベルなしデータの中でラベル付けを行うデータの抽出方法について述べる.
\ref{sec:提案手法アルゴリズム}節では,提案手法アルゴリズムを示す.
提案モデルではSemiBoostを二値分類から多値分類へ拡張する手法を提案し,分類精度の向上を図る.
%%%%%%%%%%%%%%%%%%%%%%%%%%%%%%%%%%%%%%%%%
\section{提案への着想}
%%%%%%%%%%%%%%%%%%%%%%%%%%%%%%%%%%%%%%%%%
\label{sec:提案への着想}
SemiBoostは半教師あり学習の二値分類問題において有効な手法である.しかしながらこの手法は二値分類問題にのみ適用可能な手法であるため,多値分類問題への拡張が望まれる.そこで二値分類器を多値分類に拡張する手法としてECOC法が挙げられる.ECOC法の1-versus-the rest法などを用いることで多値分類へ拡張することができる.\par

1-versus-the rest法を用いてSemiBoostを多値分類へ拡張することを考えた場合,分類器の出力を確率値とすることでカテゴリを直接推定できる.しかしながら,半教師あり学習で用いる少数のラベルありデータは,何らかの行動の結果として得られているデータであり,実際のデータ分布に従ったランダムサンプリングではない場合がある.このときSemiBoostを1-versus-the rest法に直接適用すると,異なるカテゴリを二つにまとめて二値分類するため,分類する二つの分布が特徴空間上離れた位置関係ではなくなり,SemiBoostが上手く機能せず分類精度が悪化する可能性がある.これは,正しく仮ラベルが付与されるラベルなしデータが少なくなり,誤った仮ラベルの付与されたデータの割合が多くなることに起因する.\par
\newpage
そこで本研究ではECOC法にSemiBoostを適用し,ラベルなしデータに付与された仮ラベルの信頼性が高いデータのみを仮ラベルありデータとして抽出する.そしてラベルありデータと信頼度の高い仮ラベルありデータを学習データとして教師あり学習を行う.提案手法では,ECOC法にSemiBoostを適用し,ラベルなしデータのうち信頼性の高いデータのみを抽出するため,ECOC法を直接適用したときの問題点を克服し,多値分類への拡張を行うことが可能となる.\par


%%%%%%%%%%%%%%%%%%%%%%%%%%%%%%%%%%%%%%%%%
\section{ラベルなしデータの仮ラベル付け}
%%%%%%%%%%%%%%%%%%%%%%%%%%%%%%%%%%%%%%%%%
\label{sec:ラベル付け}


提案手法では,ECOC法の一つである1-versus-the rest法を用いて多値分類を行う.あるカテゴリの符号語$\bm{W}_{c_k}$とSemiBoostによる二値分類の出力ベクトル$\bm{g}$のハミング距離が$0$となるラベルなしデータのみにカテゴリ$c_k$のラベルを付与し,このデータを仮ラベルありデータとする.その結果,複数のカテゴリに所属すると判断された信頼度の低いデータは誤った分類を引き起こす要因となる可能性があるため取り除き,信頼度の高いラベルのみを学習データに追加することが可能となる.

\begin{figure}[H]
\begin{center}
表4.1:符号表の例($K = 4$)
\label{4cate}
\includegraphics[width=10.0cm]{1vs.png}
\end{center}
\end{figure}\par

表4.1を用いて例を挙げると,SemiBoostによる二値分類の出力ベクトル$\bm{g}=({1}, {-1}, {-1}, {-1})$のようなときは$c_1$とのハミング距離が0となるため$c_1$としてラベルを付与する.一方,$\bm{g}=({1}, {-1}, {1}, {-1})$のときはハミング距離が0となるカテゴリが存在せず,最小距離復号を行っても$c_1, c_3$のどちらに所属するか判断できないので,このデータにラベルを付与しない.\par


ラベルなしデータにラベルを付与することで,学習に使用するデータの数が増加し,教師あり学習の分類精度の向上が期待される.\par


%%%%%%%%%%%%%%%%%%%%%%%%%%%%%%%%%%%%%%%%%
\section{提案手法アルゴリズム}
%%%%%%%%%%%%%%%%%%%%%%%%%%%%%%%%%%%%%%%%%
\label{sec:提案手法アルゴリズム}

\begin{description}
  \item[Step1)] 学習データを用い,1-vs-the rest法に従って構成された各二値分類器を,SemiBoostにより学習する.

  \item[Step2)]Step1の出力結果から,ラベルなしデータと各カテゴリとのハミング距離を算出する.

  \item[Step3)]ラベルなしデータのうち,ハミング距離が$0$となるカテゴリをもつデータにそのカテゴリを付与して仮ラベルありデータとする.

  \item[Step4)] Step3で得られた仮ラベルありデータとラベルありデータを用いて,教師あり学習を行い新規入力データの所属カテゴリを推定する.
\end{description}

%\newpage
上記の提案アルゴリズムを図を用いて説明する.\par
図\ref{bunpu1}のようなデータの分布を仮定して説明する.この分布は各カテゴリにおいてラベルを付与したデータが一部の区間にのみ存在するような問題設定を考える.例では,カテゴリ数を3とする.
\begin{figure}[H]
\centering
\includegraphics[width=10.0cm]{bunpu1.png}
\caption{データ分布}
\label{bunpu1}
\end{figure}\par
\newpage
まず,Step1において1-vs-the rest法に従ってSemiBoostにより分類を行う.ここで各分類器の識別境界は図\ref{bunpu2}のようになるとする.識別境界はカテゴリ数分存在し,例では,3本境界が得られる.
\begin{figure}[H]
\centering
\includegraphics[width=10.0cm]{bunpu2.png}
\caption{Step1の識別境界}
\label{bunpu2}
\end{figure}\par

%\newpage
図\ref{bunpu2}のような識別境界をもとに所属カテゴリを確率値として出力し,復号すると図\ref{bunpu3}のような識別境界が得られる.

\begin{figure}[H]
\centering
\includegraphics[width=10.0cm]{bunpu3.png}
\caption{直接多値分類に拡張した場合の分類境界}
\label{bunpu3}
\end{figure}\par

しかしながら,SemiBoostは複数の分布をまとめたとき,複数のカテゴリをまとめたデータ分布が複雑になり,良い識別境界を得られないため,直接多値分類に拡張した場合の識別境界は好ましくない.\par

そこでStep3により1-vs-the rest法の各分類器の結果をハミング距離によって,確実にいずれかのカテゴリに所属すると判断されるものを見つけ出し,このようなラベルなしデータのみに新たにラベルの付与を行う.その結果を図\ref{bunpu4}に示す.

\begin{figure}[H]
\centering
\includegraphics[width=10.0cm]{bunpu4.png}
\caption{ラベルなしデータへのラベルの付与}
\label{bunpu4}
\end{figure}\par

%\newpage
最後にStep4で,Step3において付与したラベルを用いて教師あり分類を行う.このときの分類結果を図\ref{bunpu6}に示す.このときの分類境界は直接分類したときと比較して,分類しやすいカテゴリの分類精度が上がると考えられる.

\begin{figure}[H]
\centering
\includegraphics[width=10.0cm]{bunpu6.png}
\caption{Step4による分類結果}
\label{bunpu6}
\end{figure}





 % 提案モデル
\chapter{評価実験}

本章では,提案モデルの有用性を検証するため,人工データを生成しモデルの検証を行った.
また,UCI機械学習レポジトリのデータを提案モデルに適用し実際のデータによる検証も行った.
%\newpage
\section{人工データを用いた実験}
%%%%%%%%%%%%%%%%%%%%%%%%%%%%%%%%%%%%%%%%%
\subsection{人工データを用いた実験及び実験条件}
%%%%%%%%%%%%%%%%%%%%%%%%%%%%%%%%%%%%%%%%%
\label{sec:人工データを用いた実験及び実験条件}
人工データを用いて提案手法の有効性を示す.\par

本実験で生成するデータのカテゴリ数は4,データの次元数$d = 12$とする.半教師あり学習で用いる少数のラベルありデータは,何らかの行動の結果として得られているデータであり,実際のデータ分布に従ったランダムサンプリングではない場合がある.そのため,サンプリングしたデータを偏らせることを目的とし,各カテゴリごとに5つの正規分布を連ねて作成する.\par
$c_1$は以下の5つの重心ベクトルから分散1の正規分布に従う乱数を生成する.\par
\begin{center}
表5.1.人工データの重心ベクトル($c_1$)\\
\scalebox{0.75}[0.8]{
\begin{tabular}{|c|c|}
\hline
&重心ベクトル\\
\hline\hline
1つ目&$(-3,-3,-3,-3,-3,-2,-2,-2,-2,-2)$\\
\hline
2つ目&$(-3,-3,-3,-3,-3,-1,-1,-1,-1,-1)$\\
\hline
3つ目&$(-3,-3,-3,-3,-3, 0, 0, 0, 0, 0)$\\
\hline
4つ目&$(-3,-3,-3,-3,-3, 1, 1, 1, 1, 1)$\\
\hline
5つ目&$(-3,-3,-3,-3,-3, 2, 2, 2, 2, 2)$\\
\hline
\end{tabular}
}
\end{center}\par

\newpage
$c_2$は以下の5つの重心ベクトルから分散1の正規分布に従う乱数を生成する.\par

\begin{center}
表5.2.人工データの重心ベクトル($c_2$)\\
\scalebox{0.75}[0.8]{
\begin{tabular}{|c|c|}
\hline
&重心ベクトル\\
\hline\hline
1つ目&$(-1,-1,-1,-1,-1,-2,-2,-2,-2,-2)$\\
\hline
2つ目&$(-1,-1,-1,-1,-1,-1,-1,-1,-1,-1)$\\
\hline
3つ目&$(-1,-1,-1,-1,-1,0,0,0,0,0)$\\
\hline
4つ目&$(-1,-1,-1,-1,-1,1,1,1,1,1)$\\
\hline
5つ目&$(-1,-1,-1,-1,-1,2,2,2,2,2)$\\
\hline
\end{tabular}
}
\end{center}\par

$c_3$は以下の5つの重心ベクトルから分散1の正規分布に従う乱数を生成する.\par

\begin{center}
表5.3.人工データの重心ベクトル($c_3$)\\
\scalebox{0.75}[0.8]{
\begin{tabular}{|c|c|}
\hline
&重心ベクトル\\
\hline\hline
1つ目&$(1,1,1,1,1,-2,-2,-2,-2,-2)$\\
\hline
2つ目&$(1,1,1,1,1,-1,-1,-1,-1,-1)$\\
\hline
3つ目&$(1,1,1,1,1,0,0,0,0,0)$\\
\hline
4つ目&$(1,1,1,1,1,1,1,1,1,1)$\\
\hline
5つ目&$(1,1,1,1,1,2,2,2,2,2)$\\
\hline
\end{tabular}
}
\end{center}\par

$c_4$は以下の5つの重心ベクトルから分散1の正規分布に従う乱数を生成する.\par

\begin{center}
表5.4.人工データの重心ベクトル($c_4$)\\
\scalebox{0.75}[0.8]{
\begin{tabular}{|c|c|}
\hline
&重心ベクトル\\
\hline\hline
1つ目&$(3,3,3,3,3,-2,-2,-2,-2,-2)$\\
\hline
2つ目&$(3,3,3,3,3,-1,-1,-1,-1,-1)$\\
\hline
3つ目&$(3,3,3,3,3, 0, 0, 0, 0, 0)$\\
\hline
4つ目&$(3,3,3,3,3, 1, 1, 1, 1, 1)$\\
\hline
5つ目&$(3,3,3,3,3, 2, 2, 2, 2, 2)$\\
\hline
\end{tabular}
}
\end{center}\par


\begin{figure}[H]
\centering
\includegraphics[width=9.0cm]{seisei1.png}
\caption{生成データのイメージ図}
\label{seisei}
\end{figure}\par

学習データは各カテゴリで1000個ずつデータを生成し,全データの生成数を$N=4000$個とする.生成データのイメージを図\ref{seisei}に示す.ラベルありデータは各カテゴリ内の正規分布いずれか一つから3個ずつ抽出するものとし,ラベルありデータの数を$N_L=12(3×4)$個とする.同様にテストデータを400個生成し,式(\ref{eq:eq15})の分類誤り率を10回算出し,その平均を分類精度として評価する.各実験でラベルありデータを再度抽出するものとした.

\begin{equation}
\label{eq:eq15}
分類誤り率 = 1-\frac{正しく分類したテストデータ数}{テストデータ数}
\end{equation}\par
人工データの概要を表5.5に示す.\par


\begin{center}
表5.5.人工データ概要\\
\scalebox{0.75}[0.8]{
\begin{tabular}{c|r|r|r|r|r|}
\hline
&カテゴリ数&次元数&ラベルなしデータ数&ラベルありデータ数&テストデータ数\\
\hline\hline
人工データ&4&10&3988&12 (3×4カテゴリ)&400\\
\hline
\end{tabular}
}
\end{center}\par

提案手法では,$K=4$の1-versus-the rest法を用いる.提案手法と比較手法で用いる教師あり分類器はRandomForests (RF), 1対他SVM とする.比較手法としてラベルありデータのみを用いた教師あり分類(ラベルありのみ),直接SemiBoostをECOC法に適用した分類(直接分類した場合)を用いた.また下限値として,ラベルありとラベルなしデータ全てにラベルが振られたときの教師あり分類(全データラベルあり)の分類誤り率も示す.また,SemiBoostの弱分類器には決定木を用いた.


%%%%%%%%%%%%%%%%%%%%%%%%%%%%%%%%%%%%%%%%%
\subsection{実験結果}
%%%%%%%%%%%%%%%%%%%%%%%%%%%%%%%%%%%%%%%%%
\label{sec:人工データを用いた実験結果}
以下の実験では各カテゴリでサンプルを抽出した分布の番号とそのときの実験結果を示す.

%%%%%%%%%%%%%%%%%%%%%%%%%%%%%%%%%%%%%%%%%
\begin{flushleft}
\large{\bfseries {分布1,6,11,16から抽出したときの実験結果}}
\end{flushleft}
%%%%%%%%%%%%%%%%%%%%%%%%%%%%%%%%%%%%%%%%%
%\subsection{分布1,6,11,16から抽出したときの実験結果}
このときの分類誤り率を以下に示す.

\begin{figure}[H]
\centering
\includegraphics[width=10.0cm]{01061116rf.png}
\caption{分布1,6,11,16から抽出したときの分類誤り率(RandomForests)}
\end{figure}\par

\begin{figure}[H]
\centering
\includegraphics[width=10.0cm]{01061116svm.png}
\caption{分布1,6,11,16から抽出したときの分類誤り率(SVM)}
\end{figure}\par

ラベルありデータの分布が各カテゴリで全部同じ方向に偏っているときは,ラベルありデータのみで分類したときも良い精度になり提案手法でも同程度の分類精度になった.また,直接SemiBoostを用いて分類した場合,精度が悪化しており,SemiBoostの直接的な多値分類が難しいことがわかる.

%%%%%%%%%%%%%%%%%%%%%%%%%%%%%%%%%%%%%%%%%
\begin{flushleft}
\large{\bfseries {分布1,10,11,20から抽出したときの実験結果}}
\end{flushleft}
%%%%%%%%%%%%%%%%%%%%%%%%%%%%%%%%%%%%%%%%%

%\subsection{分布1,10,11,20から抽出したときの実験結果}
このときの分類誤り率を以下に示す.
\begin{figure}[H]
\centering
\includegraphics[width=10.0cm]{01101120rf.png}
\caption{分布1,10,11,20から抽出したときの分類誤り率(RandomForests)}
\end{figure}\par

\begin{figure}[H]
\centering
\includegraphics[width=10.0cm]{01101120svm.png}
\caption{分布1,10,11,20から抽出したときの分類誤り率(SVM)}
\end{figure}\par

分布が各分布で交互に外側に存在する場合,提案手法の有効性が示された.これはラベルありデータのみでは,ラベルありデータが少ないため精度が上がらない.しかしながら提案手法ではSemiBoostを用いたのちラベルを付与したデータが正しく増加したため精度が改善されたと考えられる.\par
この人工データセットにおいて提案手法の教師あり分類器はSVMよりRandomForestsの方が精度がよい.原因としてはデータセットの特徴量をRandomForestsが容易に重要な特徴量としてとらえやすいことが原因であると考えられる.

\section{UCIデータセットを用いた実験}
%%%%%%%%%%%%%%%%%%%%%%%%%%%%%%%%%%%%%%%%%
\subsection{実験データ及び実験条件}
%%%%%%%%%%%%%%%%%%%%%%%%%%%%%%%%%%%%%%%%%
\label{sec:実験条件}

データセットとして,UCI機械学習レポジトリ\cite{UCI}の drug consumption から,CannabisとNicotineの2種類を用いる.データセットのカテゴリ数は7,データの次元数は$d = 12$である.1885件のデータのうち$N_L=210(30×7)$個はラベルありデータ,$N_U=1675$件はラベルなしデータとする.ラベルありデータは5.2.2節の方法より$N_L$個抽出し,各実験でラベルありデータを再度抽出するものとした.実験結果は5分割交差検定を5回繰り返し行い,その平均を用いることとした.
提案手法では,$K=7$の1-versus-the rest法を用いる.提案手法と比較手法で用いる教師あり分類器はRandomForests (RF), 1対他SVM とする.比較手法としてラベルありデータのみを用いた教師あり分類(ラベルありのみ),直接SemiBoostをECOC法に適用した分類(直接分類した場合)を用いた.また下限値としてラベルありとラベルなしデータ全てにラベルが振られたときの教師あり分類(全データラベルあり),上限値としてランダムにデータを分類したとき(ランダム)の分類誤り率も示す.
%初期のラベルありデータ数を各カテゴリ30個全部ラベルあり
事前実験により,SemiBoostの分類器数$T=10$,ラベルありデータ数$N_L=210 (30\times7)$個,偏りのあるラベルありデータ割合を$\theta\%$とした.また,SemiBoostの分類器にはSVMを用いた.評価指標として人工と同様に分類誤り率を用いる.

%また,教師あり分類に用いる学習器はRandomForest, 1対他SVM とした.

%%%%%%%%%%%%%%%%%%%%%%%%%%%%%%%%%%%%%%%%%
\subsection{ラベルありデータの抽出}
%%%%%%%%%%%%%%%%%%%%%%%%%%%%%%%%%%%%%%%%%
\label{sec:ラベルありデータの抽出}

本研究ではラベルありデータに偏りがある場合についても,提案手法の有効性を示すことを目的とする.そのためラベルありデータは,各カテゴリの重心ベクトルを求め,そこから最も遠い点を基準に,その近傍ラベルありデータのうち$\theta \%$の点をサンプリングすることで生成する.その後,ラベルありデータのうち残りの$(100 - \theta) \%$の点はランダムに抽出し,これらをラベルありデータとする.

\begin{figure}[H]
\centering
\includegraphics[width=10.0cm]{label.png}
\caption{ラベルありデータのサンプル法}
\label{label}
\end{figure}\par


%%%%%%%%%%%%%%%%%%%%%%%%%%%%%%%%%%%%%%%%%
\subsection{実験結果}
%%%%%%%%%%%%%%%%%%%%%%%%%%%%%%%%%%%%%%%%%
\label{sec:実験結果}
%%%%%%%%%%%%%%%%%%%%%%%%%%%%%%%%%%%%%%%%%
\begin{flushleft}
\large{\bfseries {$\theta=70\%$のときの実験結果}}
\end{flushleft}
%%%%%%%%%%%%%%%%%%%%%%%%%%%%%%%%%%%%%%%%%

%\subsection{$\theta=70\%$のときの実験結果}
偏りがあるデータの割合$\theta=70\%$のとき,各データセットにおける分類誤り率を以下に示す.


\begin{figure}[H]
\centering
\includegraphics[width=10.0cm]{canarf.png}
\caption{Canabisによる分類結果($\theta=70\%$)(RandomForests)}
\label{canarf}
\end{figure}\par
\begin{figure}[H]
\centering
\includegraphics[width=10.0cm]{canasvm.png}
\caption{Canabisによる分類結果($\theta=70\%$)(SVM)}
\label{canasvm}
\end{figure}\par
\begin{figure}[H]
\centering
\includegraphics[width=10.0cm]{nicorf.png}
\caption{Nicotineによる分類結果($\theta=70\%$)(RandomForests)}
\label{nicorf}
\end{figure}\par
\begin{figure}[H]
\centering
\includegraphics[width=10.0cm]{nicosvm.png}
\caption{Canabisによる分類結果($\theta=70\%$)(SVM)}
\label{nicosvm}
\end{figure}\par


図\ref{canarf},図\ref{canasvm},図\ref{nicorf},図\ref{nicosvm}よりどちらのデータセットにおいても提案手法がラベルありデータのみを用いた教師あり分類,直接SemiBoostをECOC法に適用した分類より誤分類率が小さくなった.この結果により,仮ラベルを付与してから再度教師あり分類を行う提案手法の有効性が示された.\par


このとき,ラベルを付与したデータの数と付与したラベルの正解数合計と不正解数合計を以下の表に示す.

\begin{center}
表5.6.ラベル付け結果概要(Canabis)\\
\scalebox{0.75}[0.8]{
\begin{tabular}{c|r|r|r|r|r|r|r}
\hline
&1&2&3&4&5&6&7\\
\hline\hline
正しく付与したラベル&602&19&4&9&3&0&490\\
\hline
誤って付与したラベル&370&5&8&14&1&1&286\\
\hline
\end{tabular}
}
\end{center}\par
\begin{center}
表5.7.ラベル付け結果概要(Nicotine)\\
\scalebox{0.75}[0.8]{
\begin{tabular}{c|r|r|r|r|r|r|r}
\hline
&1&2&3&4&5&6&7\\
\hline\hline
正しく付与したラベル&1&23&16&4&0&82&155\\
\hline
誤って付与したラベル&1&8&30&1&0&199&15\\
\hline
\end{tabular}
}
\end{center}\par
\vspace{1.0cm}
これらの結果から付与したラベルが真のラベルと一致している割合がCanabisにおいて$62\%$,Nicotineにおいて$53\%$程度となった.この結果はラベル付けの正確さを示しており,この正解率は全てのデータにラベルを付与して分類を行った際の正解率以上となっているため,提案手法が有効であると考えられる.しかしながらラベルを付与したデータ数がカテゴリによって差があり,分類しやすいデータにはラベルが付与されやすく分類精度も良い傾向が見られた.
%\newpage
%%%%%%%%%%%%%%%%%%%%%%%%%%%%%%%%%%%%%%%%%
\begin{flushleft}
\large{\bfseries {$\theta$を変更したときの実験結果}}
\end{flushleft}
%%%%%%%%%%%%%%%%%%%%%%%%%%%%%%%%%%%%%%%%%

%\subsection{$\theta$を変更したときの実験結果}

次にラベルありデータが全て偏っている場合$\theta=100\%$,完全にランダムにサンプリングした場合$\theta=0\%$についても\ref{sec:実験条件}節の条件で実験を行う.\par
偏りがあるデータの割合$\theta=100\%$のとき,各データセットにおける分類誤り率を以下に示す.

\begin{figure}[H]
\centering
\includegraphics[width=10.0cm]{canabis1.png}
\caption{Canabisによる分類結果($\theta=100\%$)(RandomForests)}
\label{canabis1}
\end{figure}\par
\begin{figure}[H]
\centering
\includegraphics[width=10.0cm]{nicotine2.png}
\caption{Canabisによる分類結果($\theta=100\%$)(SVM)}
\label{canabis2}
\end{figure}\par



\begin{figure}[H]
\centering
\includegraphics[width=10.0cm]{nicotine1.png}
\caption{Nicotineによる分類結果($\theta=100\%$)(RandomForests)}
\label{nicotine1}
\end{figure}\par
\begin{figure}[H]
\centering
\includegraphics[width=10.0cm]{nicotine2.png}
\caption{Nicotineによる分類結果($\theta=100\%$)(SVM)}
\label{nicotine2}
\end{figure}\par

$\theta=100\%$として全てのデータを偏らせたとき,提案手法で教師あり分類器としてRandomForestsを使っているものは,ラベルありデータのみを用いた教師あり分類,直接SemiBoostをECOC法に適用した分類より誤分類率が小さくなった.ラベルありデータが偏った場合でも提案手法が有効であることが分かる.
しかしながら,提案手法でも教師あり分類器にSVMを使ったものは直接SemiBoostをECOC法に適用した分類より分類誤り率が大きくなった.これは偏りのあるデータの中にSVMのサポートベクトルとなる点があり,ラベルありデータのみのSVMと同様に好ましくない識別境界が引かれていると考えられる.\par
よってこの実験から偏りの大きなデータを多数含んでいる可能性がある場合,提案手法の教師あり分類器としてSVMが好ましくない可能性があると考えられる.\par

%\newpage
次に完全にランダムにサンプリングした場合$\theta=0\%$の,各データセットにおける分類誤り率を以下に示す.\par

\begin{figure}[H]
\centering
\includegraphics[width=10.0cm]{canabis3.png}
\caption{Canabisによる分類結果($\theta=0\%$)(RandomForests)}
\label{canabis1}
\end{figure}\par
\begin{figure}[H]
\centering
\includegraphics[width=10.0cm]{nicotine4.png}
\caption{Canabisによる分類結果($\theta=0\%$)(SVM)}
\label{canabis2}
\end{figure}\par


\begin{figure}[H]
\centering
\includegraphics[width=10.0cm]{nicotine3.png}
\caption{Nicotineによる分類結果($\theta=0\%$)(RandomForests)}
\label{nicotine1}
\end{figure}\par
\begin{figure}[H]
\centering
\includegraphics[width=10.0cm]{nicotine4.png}
\caption{Nicotineによる分類結果($\theta=0\%$)(SVM)}
\label{nicotine2}
\end{figure}\par

$\theta=0\%$としてランダムサンプリングを行った場合,提案手法ではラベルありデータのみで分類した場合と同程度の精度を示した.直接SemiBoostをECOC法に適用した分類の精度がラベルありデータのみで分類した場合より分類誤り率が大きいことからECOC法を直接SemiBoostに適用することは好ましくないことがわかる.しかしながら,提案手法ではこの精度が悪化する識別境界の影響を受けにくく,ラベルありデータのみで分類を行った場合の分類誤り率と同程度になった.\par

これらの結果から,提案手法は$\theta$の値が大きい偏りのあるデータに対して強く,$\theta$の値が小さい偏りのないデータのときにも精度が悪化しないことが示された.






 % 実験
\chapter{考察}
本章では,本研究全体を通した考察を述べる.
\section{分類精度}
%\begin{flushleft}
%  \large{\bfseries {}}
%\end{flushleft}\par
第5章の結果より,偏りのあるデータに対して提案手法の有効性が示された.直接多値分類を行うことによる分類器の精度の悪化を引き起こすSemiBoostの欠点を打ち消しつつ,SemiBoostの偏りのあるデータに対して強い性質を活用することができた.その結果,半教師あり学習の問題点であるラベルありデータの偏りに影響をされない分類を行う提案の有効性を示すことができた.\par
また,人工データの実験結果から偏りのあるデータの位置によって分類の難しさが変化することが分かった.提案手法において,比較的分類しやすいカテゴリの分類がされやすくなっている.これはラベル付けされたデータが多いカテゴリに見られる傾向であり,その結果が分類精度に影響したと考えられる.\par

実験全体を通して偏りのあるデータを扱う際,SVMが分類精度に大きく影響しているような傾向が見られた.これは実験を行う際のデータの適正が考えられるため,SVMによる分類と決定木(本研究では,RandomForest)による分類のどちらが良いかは分からないのでデータに合わせて分類器を決める必要があると考えられる.特徴量に依存しそうである場合には,決定木の方が精度が良いと考えられる.SVMは,ラベルを付与したものに誤分類したものがあると精度が悪化する可能性がある.\par


\section{ラベルの付与について}
ラベルなしデータに対して付与するラベルの正確さは分類の難しさによって変化するため,ハミング距離が0となるデータのみを抽出する復号方法を用いても正確なラベルの付与が行えなかった.また,この手法で復号を行うとカテゴリ数が増えるほどハミング距離が0となるラベルを付与できるデータが少なくなるため,カテゴリ数が多いときにはこの手法の効果が無くなることがある.\par
ラベルを付与しやすいカテゴリとそうでないカテゴリが存在し,全データのうち他のカテゴリと距離が遠いカテゴリほどラベルを付与しやすいことが分かった.一方で,他のカテゴリと距離が近いデータにはラベルを付与しにくい傾向があり,複数のカテゴリをまとめて分類を行う提案手法では改善が難しいと考えられる.\par
SemiBoostにおいて,仮ラベルから学習データに用いるデータを経験的に$10\%$としていたが,分類器の作成において学習に用いるデータ数を検討するべきである.これは弱学習器の学習が複雑になるため,少しでも正しい分類を行えるようにしたい意図がある.\par

\section{計算コスト}
提案手法ではSemiBoostを用いたラベルの付与を行う際,カテゴリ数と同数の1-versus-the rest法の判別器で学習を行う必要がある.このときの学習は分散処理を行うことができるため,提案手法での学習時間はSemiBoostによる二つにまとめたカテゴリの分類時間と,その後の教師あり分類器の学習時間となる.したがって,カテゴリ数が増加したとしても学習時間に影響は少なく,全データ数の学習により計算時間が決まると考えられる.この手法において計算量が多い点はSemiBoostのデータ間類似度を測定する部分が主な点である.これはグラフベースな手法の問題点でもあり,計算量の削減も重要な課題である.

 % 全体考察
\chapter{結論と今後の課題}

%%%%%%%%%%%%%%%%%%%%%%%%%%%%%%%%%%%%%%%%%
\section{結論}
%%%%%%%%%%%%%%%%%%%%%%%%%%%%%%%%%%%%%%%%%
\label{sec:結論}
本研究では,SemiBoostの偏りのあるデータに強い性質を利用し,多値分類問題に拡張することでSemiBoostの多値分類への拡張モデルを提案した.その際,直接ECOC法にSemiBoostを適用した場合に良い識別境界を得ることができない可能性がある.その問題点を解決するため,SemiBoostによってラベルなしデータに与えた仮ラベルの信頼度の評価を行い,その信頼度の高いもののみにラベルを付与した.このラベルを付与したデータとラベルありデータを教師あり学習に用いることで,識別境界の悪化を避けつつ,SemiBoostによってラベルを付与したデータによりラベルありデータのみの場合と比較した精度の改善を図った.\par
また,UCIデータセットを用いることで偏りのあるデータに対して提案モデルを適用する有効性を示した.また,ラベルの付与しやすいデータと付与しづらいデータから分類の難しさも分かった.偏りのないランダムサンプルを行ったデータで実験を行った場合,ラベルありデータと同程度の精度を示すことから,データの偏りに関わらず提案手法を利用できることを示すことができた.本研究の成果により,多値分類問題に対しても,精度の良い半教師ありブースティング学習が可能となり,より良い対象問題への応用が期待できる.\par

%%%%%%%%%%%%%%%%%%%%%%%%%%%%%%%%%%%%%%%%%
\section{今後の課題}
%%%%%%%%%%%%%%%%%%%%%%%%%%%%%%%%%%%%%%%%%
\label{sec:今後の課題}
今後の課題として,以下の2点があげられる.\par
全てのデータにラベルがある場合の精度へ近づけることが考えられる.これはラベルなしデータにラベルを付与する精度を上げることとラベルを付与するデータを増やすことが考えられる.\par
カテゴリ数が未知の場合にも対応した半教師あり多値分類手法への拡張などが挙げられる.今回の提案手法ではECOC法を用いたがカテゴリ数が未知の場合は符号表を用いることができなくなるため,全く異なる提案が必要になる.分類方法としては,クラスタリングを利用しつつ分類を行うことで,あるカテゴリに所属するか否かを判断するモデルの作成が考えられる.\par
 % まとめと今後の課題
\chapter*{謝辞}
\addcontentsline{toc}{chapter}{謝辞}
\markright{謝辞}

本研究を行うに当たり早稲田大学創造理工学部経営システム工学科教授・後藤正幸先生にはご多忙の中,並々ならぬ御尽力,御指導を賜りました.ここに深く感謝の意を表します.

湘南工科大学工学部情報工学科講師・三川健太先生には,サブゼミから卒業論文執筆に至るまで,多大なるご尽力を賜りました.特に,本研究の実験,論文の構成,数式の正しい表記の仕方など,数多くのご助言,熱心なご指導を賜りました.ここに深く感謝致します.

上智大学理工学部助教授・山下遥先生には,合宿での発表の際に多くの御助言を賜りました.また,研究以外にも様々なアドバイスを賜りました.ここに深く感謝の意を表します.

早稲田大学大学院創造理工学研究科博士課程3年・楊添翔氏,早稲田大学大学院創造理工学研究科博士課程2年・雲居玄道氏,蓮本恭輔氏,早稲田大学大学院創造理工学研究科博士課程1年・鈴木佐俊氏にも毎週のゼミに加えて,合宿での発表の際に多くの御助言を賜りました.また,研究以外にも様々なアドバイスを賜りました.ここに深く感謝の意を表します.

また,早稲田大学大学院創造理工学研究科修士課程2年 中野修平氏,西尾友里氏,早稲田大学大学院創造理工学研究科修士課程1年 杉崎智哉氏,安井一貴氏にも,サブゼミから卒業論文執筆に至るまで,数多くの有意義なご助言を賜りました.また,研究以外にも研究室生活の過ごし方などについても多くの御助言を賜りました.ここに感謝の意を表します.

さらに,早稲田大学大学院創造理工学研究科修士課程2年 大窪啓介氏,河部瞭太氏,清水良太郎氏,杉山祐貴氏,水落洋貴氏,張笑エン氏,早稲田大学大学院創造理工学研究科修士課程1年 新井浩健氏,井上一磨氏,大川順也氏,大堀祐一氏,金澤真平氏,杉崎智哉氏,世古裕都氏,藤波英輝氏,保坂大樹氏,保戸田未桜氏,にも,毎週のゼミにおいて,数多くのご助言を賜りました.ここに感謝の意を表します.

後藤研究室学部4年 桑田和さん,後藤亮介君,齋藤陽佑君,西村祐樹君,福山武士君,松元琢真君,山之内薫さん,西口智之君,服部達也君には,サブゼミでの研究に対する助言や研究室での生活など,公私共に多くの時間を共有し大変有意義なものにして頂き深く感謝致します. 加えて,山手学院高等学校3年舟田清香さんには,心身共に大変お世話になりました.ここに感謝の意を表します.

最後に,お世話になった方々,大学生活を見守ってくれた両親,多くの支えになってくれた友人に心から感謝致します.

 \\
\hspace*{15zw}平成31年2月1日

\hspace*{15zw}早稲田大学 創造理工学部

\hspace*{15zw}経営システム工学科 

\hspace*{15zw}情報数理応用研究 後藤研究室

\hspace*{15zw}1X15C048-2
\begin{thebibliography}{99} 
\addcontentsline{toc}{chapter}{参考文献}
\markright{参考文献}
%%%%%%%%%%%%%%%%%%%%%%%%%
% 参考文献
%%%%%%%%%%%%%%%%%%%%%%%%%
\bibliographystyle{junsrt}
\bibliography{references}
\end{thebibliography}

%\appendix
% \chapter*{付録}
% \addcontentsline{toc}{chapter}{付録}
% \markright{付録}
% \renewcommand{\baselinestretch}{0.8}

\chapter{潜在クラス数の決定}

%%%%%%%%%%%%%%%%%%%%%%%%%%%%%%%%%%%%%%%%%
\section{菓子カテゴリの潜在クラス数決定}
%%%%%%%%%%%%%%%%%%%%%%%%%%%%%%%%%%%%%%%%%
GMMによる潜在クラス数の決定には,赤池情報量規準(AIC) \cite{AIC} を用いる.潜在クラス数を変化させてGMMを行い,AICを求める.AICが収束した時の値を潜在クラス数とした.\par
以下に,菓子カテゴリの周期成分とイベント成分の潜在クラス数を変化させたときのAIC値の推移を示す.周期成分とイベント成分のAICはそれぞれ以下の式(\ref{eq:AIC_eq1})(\ref{eq:AIC_eq2})で表される.

\begin{eqnarray}
\label{eq:AIC_eq1}
 AIC = - 2 ln\frac{\pi_{j} \mathcal{N}({\bm x}_n|{\bm \mu}_{j},{\bm \Omega}_{j})}{\sum^{J}_{j=1}\pi_{j} \mathcal{N}({\bm x}_n|{\bm \mu}_{j},{\bm \Omega}_{j})} + 54,446
\end{eqnarray}

\begin{eqnarray}
\label{eq:AIC_eq2}
 AIC = - 2 ln\frac{\pi_{l} \mathcal{N}({\bm x}_n|{\bm \nu}_{l},{\bm \Gamma}_{l})}{\sum^{L}_{l=1}\pi_{l} \mathcal{N}({\bm x}_n|{\bm \nu}_{l},{\bm \Gamma}_{l})} + 54,446
\end{eqnarray}

\begin{center}
\begin{figurehere}
\includegraphics[scale=0.8]{aic_kashi_week_s.png}
\caption{菓子カテゴリ(周期成分)のAIC値の推移}
\label{fig: aic_kashi_week}
\end{figurehere}
\end{center}
\begin{center}
\begin{figurehere}
\includegraphics[scale=0.8]{aic_kashi_event_s.png}
\caption{菓子カテゴリ(イベント成分)のAIC値の推移}
\label{fig: aic_kashi_event}
\end{figurehere}
\end{center}\par

\newpage

図\ref{fig: aic_kashi_week}より,潜在クラス数が7のときからAIC値の減少が緩やかになっている.よって,周期成分の潜在クラス数は7とした.同様に,図\ref{fig: aic_kashi_event}より,潜在クラス数が7のときからAIC値の減少が緩やかになっている.よって,周期成分の潜在クラス数は7とした.

%%%%%%%%%%%%%%%%%%%%%%%%%%%%%%%%%%%%%%%%%
\section{酒カテゴリの潜在クラス数決定}
%%%%%%%%%%%%%%%%%%%%%%%%%%%%%%%%%%%%%%%%%
酒カテゴリも菓子カテゴリと同様に,潜在クラス数を変化させてGMMを行い,AICを求める.AICが収束した時の値を潜在クラス数とした.\par
以下に,菓子カテゴリの周期成分とイベント成分の潜在クラス数を変化させたときのAIC値の推移を示す.周期成分とイベント成分のAICはそれぞれ以下の式(\ref{eq:AIC_eq3})(\ref{eq:AIC_eq4})で表される.

\begin{eqnarray}
\label{eq:AIC_eq3}
 AIC = - 2 ln\frac{\pi_{j} \mathcal{N}({\bm x}_n|{\bm \mu}_{j},{\bm \Omega}_{j})}{\sum^{J}_{j=1}\pi_{j} \mathcal{N}({\bm x}_n|{\bm \mu}_{j},{\bm \Omega}_{j})} + 30,210
\end{eqnarray}

\begin{eqnarray}
\label{eq:AIC_eq4}
 AIC = - 2 ln\frac{\pi_{l} \mathcal{N}({\bm x}_n|{\bm \nu}_{l},{\bm \Gamma}_{l})}{\sum^{L}_{l=1}\pi_{l} \mathcal{N}({\bm x}_n|{\bm \nu}_{l},{\bm \Gamma}_{l})} + 42,294
\end{eqnarray}

\begin{center}
\begin{figurehere}
\includegraphics[scale=0.75]{aic_alcohol_week_s.png}
\caption{酒カテゴリ(周期成分)のAIC値の推移}
\label{fig: aic_alcohol_week}
\end{figurehere}
\end{center}
\begin{center}
\begin{figurehere}
\includegraphics[scale=0.75]{aic_alcohol_event_s.png}
\caption{酒カテゴリ(イベント成分)のAIC値の推移}
\label{fig: aic_alcohol_event}
\end{figurehere}
\end{center}\par

図\ref{fig: aic_alcohol_week}より,潜在クラス数が5のときからAIC値の減少が緩やかになっている.よって,周期成分の潜在クラス数は5とした.同様に,図\ref{fig: aic_alcohol_event}より,潜在クラス数が7のときからAIC値の減少が緩やかになっている.よって,周期成分の潜在クラス数は7とした.



\end{document}